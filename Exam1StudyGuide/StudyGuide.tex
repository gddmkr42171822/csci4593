\documentclass[12pt]{article}
\linespread{1.3} % 1 and half line spacing
\usepackage[top=1in, bottom=1in, left=.5in, right=.5in]{geometry} % set the margins
\usepackage{courier} % font type
\usepackage{amsmath} % math equations

\begin{document}
	
\title{Exam 1: Study Guide}
\author{}
\date{}

\maketitle

\section{Famous Architects and Machines}

\subsection*{}
\begin{tabular}{|c|c|}
	\hline
	\textbf{Architect} & \textbf{Contribution}\\
	\hline
	Eckert and Mauchly & First Working Electronic Computer (ENIAC)(1946)\\
	\hline
	Wilkes & First Store Program Computer (EDSAC I) (1949)\\
	\hline
	Amdahl & Amdahl's Law\\
	\hline
	Cray & 1st Super Computer\\
	\hline
	Patterson & RISC Processor Design\\
	\hline
	Hennessy & MIPS\\
	\hline
	Rau & Very Long Instruction Word\\
	\hline
	Smith & Branch Prediction Strategies\\
	\hline
	Patt & First Complex Logic Gate On Silicon\\
	\hline
	Hwu & Compiler Optimization\\
	\hline
	Sohi & Non-blocking Microprocessor Caches\\
	\hline
\end{tabular}

\subsection*{}
\begin{tabular}{|c|c|}
	\hline
	\textbf{Computer} & \textbf{Claim To Fame}\\
	\hline
	Univac I & First Commercial Machine (1951)\\
	\hline
	IBM System 360 & Concept Of Family Of Machines (1964)\\
	\hline
	DEC PDP-8 & Mini-computer\\
	\hline
	DEC PDP-11 & Unix Developled On This Machine\\
	\hline
	Cray-1 & Super Computer (1976)\\
	\hline
	Intel 4004 & First Microprocessor\\
	\hline
	Intel 8086 & Basic Architecture of IA32 PC\\
	\hline
	Intel 80486 & Pipelined IA32 \\
	\hline
	Pentium & Superscalar IA32 \\
	\hline
	AMD Barcelona & L1, L2, L3 Cache; nm Technology; 1.6 GHz clock\\
	\hline
\end{tabular}

\section{Fabrication}

\subsection*{Yield}
\begin{align*}
	&\mathrm{\textbf{Cost\;Per\;Die}} = \frac{\mathrm{Cost\;Per\;Wafer}}{\mathrm{Dies\;Per\;Wafer}\cdot{\mathrm{Yield}}}\\
	&\mathrm{\textbf{Wafer\;Area}} = \pi\cdot{r}^2\\
	&\mathrm{\textbf{Dies\;Per\;Wafer}} = \frac{\mathrm{Wafer\;Area}}{\mathrm{Die\;Area}}\\
	&\mathrm{\textbf{Yield}} = \frac{1}{(1\;+\;(\mathrm{Defects\;Per\;Area\cdot{\frac{\mathrm{Die\;Area}}{2}}}))^2}\\
\end{align*}

\subsection*{Power}
\begin{align*}
	\mathrm{\textbf{Dynamic\;Power}} = \frac{1}{2}\cdot{\mathrm{Capacitive\;Load}}\cdot{\mathrm{Voltage}^2}\cdot{\mathrm{Clock\;Rate}} 
\end{align*}

\section{MIPS Design Decisions}

\subsection*{Why is MIPS limited to 32 registers?}
\begin{enumerate}
\item{A very large number of registers may increase the clock cycle time simply because it takes electronic signals longer when they must travel farther.}
\item{You would have to expand the instruction format to accomadate more bits.}
\end{enumerate}

\subsection*{What are the 3 design principles of MIPS}
\begin{enumerate}
	\item{Simplicity favors regularity.}
	\item{Smaller is faster.}
	\item{Good design demands good compromises.}
\end{enumerate}

\subsection*{Why do words in MIPS have to start at addresses that are multiples of 4?}
\begin{enumerate}
	\item{This alignment restriction leads to faster data transfers.}
	\item{It also allows programs to always use lw and sw to access the stack.}
\end{enumerate}

\subsection*{Why does MIPS keep all instructions the same length?}
\begin{enumerate}
	\item{MIPS follows the design principle that states `simplicity favors regularity.'}
\end{enumerate}


\section{Intel x86 Idiosyncrasies}
\begin{enumerate}
	\item{Intel x86 only supports 8 general purpose register.}
	\item{The x86 arithmetic and logical instructions must have one operand act as both a source and a destination.}
	\item{One instruction operand can be in memory.}
	\item{x86 instructions are not all 4 bytes in length.}
	\item{Registers have dedicated uses.}
\end{enumerate}

\section{Amdahl's Law}
\subsection*{}
\begin{align*}
	\mathrm{\textbf{Speed\;Up}} = \frac{\mathrm{Old\;Time}}{\mathrm{New\;Time}}\\
	\mathrm{\textbf{Speed\;Up}}_{\mathrm{\textbf{Overall}}} = \frac{1}{(1-\mathrm{F})+\frac{\mathrm{F}}{\mathrm{S}}}\\
\end{align*}
where F is the Percentage to be enchanced and S is the factor it is to be enhanced by\\
\section{CPI and Performance}
\subsection*{CPU Time}
\begin{align*}
		\mathrm{\textbf{CPU\;Time}} &= \frac{\mathrm{Instruction\;Count}\cdot{\mathrm{CPI}}}{\mathrm{Clock\;Rate}}\\
						   &= \frac{\mathrm{Insructions}}{\mathrm{Program}}\cdot{
						   \frac{\mathrm{Clock\;Cycles}}{\mathrm{Instructions}}}\cdot{\frac{\mathrm{Seconds}}{\mathrm{Clock\;Cycle}}}\\
		\mathrm{\textbf{Overall\;Effective\;CPI}} &= \sum_{i=1}^{n} (\mathrm{CPI}_{i}\cdot{\mathrm{IC}_{i}})\\
\end{align*}
	
\subsection*{MIPS}
\begin{align*}
		\mathrm{\textbf{MIPS}} = \frac{\mathrm{Instruction\;Count}}{\mathrm{CPU\;Time}\cdot{10^6}} \\
\end{align*}

\subsection*{Performance Ratio}
\begin{align*}
	\frac{\mathrm{performance}_\mathrm{A}}{\mathrm{performance}_\mathrm{B}} = \frac{\mathrm{CPU\;
	Time}_{\mathrm{A}}}{\mathrm{CPU\;Time}_{\mathrm{B}}} = \mathrm{n}\\
\end{align*}
\begin{center}
So A is n times faster than B
\end{center}

\subsection*{SPEC}
	\[
		\mathrm{\textbf{SPEC Ratio}} = \frac{\mathrm{Reference Time}}{\mathrm{Execution Time}}\\
	\]

\section{Counting Gate Delays}
\subsection*{Wallace Tree}
\[
	\mathrm{\textbf{Gate\;Delays}} = log_{\frac{3}{2}} \mathrm{N}
\]
where N is the number of bits in the multiplier\\
\[
	\mathrm{\textbf{Gate\;Delays}}_{\mathrm{\textbf{Booth\;Recoding}}} = log_{\frac{3}{2}} \frac{\mathrm{N}}{2}
\]
\subsection*{Ripple Carry Adder}
\[
	\mathrm{\textbf{Gate Delays}} = \text{N}\cdot{2}
\]
where N is the number of bits in the numbers being added
\subsection*{Carry Look Ahead Adder}
\begin{enumerate}
\item{Each FA is like a ripple carry adder and has 1 gate delay each.}
\item{The first-level and first CLA is 3 gate delays with an additional 2 after each CLA on the same level.}
\item{Each additional first CLA in a level is another 2 gate delays on top of the 3 for the first-level CLA.}
\end{enumerate}
\section{Binary Arithmetic}

\subsection*{Overflow}
\subsubsection*{Unsigned}
\begin{enumerate}
	\item{Overflow occurs when there is a 1 in the carry out.}
\end{enumerate}

\subsubsection*{Signed}
\begin{enumerate}
	\item{Overflow occurs when you add 2 positive numbers and get a negative, or when you add 2 negative numbers and get a positive.}
\end{enumerate}

\subsection*{Addition}
\subsubsection*{Unsigned}
\begin{enumerate}
	\item{The result should have the same amount of bits are the original numbers.}
\end{enumerate}

\subsubsection*{Signed}
\begin{enumerate}
	\item{Same as Unsigned except you have to watch out for the MSB being a 1 or 0 indicating a positive or negative number respectively.}
\end{enumerate}

\subsection*{Subtraction}
\subsubsection*{Unsigned}
\begin{enumerate}
	\item{Change the subtrahend to its 2's complement and add it to the minuend.}
\end{enumerate}

\subsubsection*{Signed}
\begin{enumerate}
	\item{Same as Unsigned.}
\end{enumerate}

\subsection*{Multiplication}
\subsubsection*{Unsigned}
\begin{enumerate}
	\item{If multiplying an N-bit number by an M-bit number, the result should have N+M bits.}
\end{enumerate}

\subsubsection*{Signed}
\begin{enumerate}
	\item{Same as Unsigned.}
	\item{If the multiplier is negative take the 2's complement on both numbers and then multiply them.}
	\item{Use Booth's recoding to reduce the number of additions needed to be done by half.}
	\item{If using Booth's recoding and the the multiplier has a -1 in the position add the 2's complement of the multiplicand to the product.}
	\item{If using Booth's recoding and the mulitipler has a -2 in the position find the 2's complement of the multiplicand and shift it left 1 bit
	and then add it to the product.}
	\item{If using Booth's recoding and the mulitipler has a 2 in the position shift the multiplicand left 1 bit
	and then add it to the product.}
\end{enumerate}

\subsection*{Division}
\subsubsection*{Unsigned}
\begin{enumerate}
	\item{Remember when borrowing from the left that the 1 really carries over 2 1's.}
\end{enumerate}

\subsubsection*{Signed}
\begin{enumerate}
	\item{Make both numbers positive and then figure out the sign at the end and change the bits in the result to reflect that.}
\end{enumerate}


\end{document}