\documentclass[12pt]{article}
\linespread{1.3} % 1 and half line spacing
\usepackage[top=1in, bottom=1in, left=.5in, right=.5in]{geometry} % set the margins
\usepackage{helvet} % font type
\usepackage{amsmath} % math equations

\begin{document}
	
\title{CSCI 4593: Homework 2}
\author{}
\date{}

\maketitle

\section*{2.3}
sub \$t0, \$s3, \$s4;\hspace{10mm}\$t0 = i - j\newline
sll \$t0, \$t0 , 2; \hspace{10mm} \$t0 = (i - j) * 4\newline
add \$t0, \$s6, \$t0;\hspace{10mm}\$t0 = \&A + (i -j)*4\newline
lw \$t1, \$0(\$t0);\hspace{11mm}\$t1 = A[i-j]\newline
sw \$t1, 32(\$s7);\hspace{12mm}B[8] = A[i-j]\newline

\section*{2.7}
\textbf{Little Endian}\\
Low Memory$\rightarrow$High Memory\\
\begin{tabular}{|c|c|c|c|c|}
	\hline
	Hex Digit & 12 & ef & cd & ab\\
	\hline
	Address & 0 & 4 & 8 & 12\\
	\hline
\end{tabular}\\
\newline
\textbf{Big Endian}\\
Low Memory$\rightarrow$High Memory\\
\begin{tabular}{|c|c|c|c|c|}
	\hline
	Hex Digit & ab & cd & ef & 12\\
	\hline
	Address & 0 & 4 & 8 & 12\\
	\hline
\end{tabular}\\

\section*{2.11}
\begin{tabular}{|c|c|c|c|c|c|c|}
	\hline
	instruction & type & opcode & rs & rt & rd & immed\\
	\hline
	addi \$t0, \$s6, 4 & I-type & 8 & 22 & 8 && 4\\
	\hline
	add \$t1, \$s6, \$0 & R-type & 0 & 22 & 0 & 9 & \\
	\hline
	sw \$t1, 0(\$t0) & I-type & 43 & 8 & 9 & & 0\\
	\hline
	lw \$t0, 0(\$t0) & I-type & 35 & 8 & 8 & & 0\\
	\hline
	add \$s0, \$t1, \$t0 & R-type & 0 & 9 & 8 & 16 & \\
	\hline
\end{tabular}

\section*{2.12.3}
\$s0 = 1000 0000 0000 0000 0000 0000 0000 0000\\
\$s1 = 1101 0000 0000 0000 0000 0000 0000 0000\\
inverted \$s1 = 0010 1111 1111 1111 1111 1111 1111 1111\\
negated \$s1 = 0011 0000 0000 0000 0000 0000 0000 0000\\
\newline
\begin{tabular}{ccccccccc}
	& 1000 & 0000 & 0000 & 0000 & 0000 & 0000 & 0000 & 0000\\
	+ & 0011 & 0000 & 0000 & 0000 & 0000 & 0000 & 0000 & 0000\\
	\hline
	& 1011 & 0000 & 0000 & 0000 & 0000 & 0000 & 0000 & 0000\\
\end{tabular}
\newline
\newline
\textbf{\$t0 = 1011 0000 0000 0000 0000 0000 0000 0000 $\rightarrow$0xB0000000}
	
\section*{2.12.4}
Yes, it is the desired result.  Converting the operands to decimal numbers and subtracting returns a negative number as does the binary subtraction.

\section*{2.14}
\begin{tabular}{|c|c|c|c|c|c|}
	\hline
	op (6 bits) & rs (5 bits) & rt (5 bits) & rd (5 bits) & shamt (5 bits) & funct (6 bits)\\
	\hline
	000000 & 10000 & 10000 & 10000 & 00000 & 100000\\
	\hline
	0 & \$s0 & \$s0 & \$s0 & 0 & 32\\
	\hline
\end{tabular}
\newline
\newline
Type = R-type\\
Assembly = add \$s0, \$s0, \$s0

\section*{2.15}
\begin{tabular}{|c|c|c|c|c|c|c|}
	\hline
	op (6 bits) & rs (5 bits) & rt (5 bits) & rd (5 bits) & shamt (5 bits) & funct (6 bits) & address/const\\
	\hline
	43 & \$t2 & \$t1 & n/a & n/a & n/a & 32\\
	\hline
	101011 & 01010 & 01001 & n/a & n/a & n/a & 0000000000100000\\
	\hline
\end{tabular}
\newline
\newline
Type = I-type\\
Binary = 1010 1101 0100 1001 0000 0000 0010 0000\\
Hex = 0xAD490020

\section*{2.17}
\begin{tabular}{|c|c|c|c|c|c|c|}
	\hline
	op (6 bits) & rs (5 bits) & rt (5 bits) & rd (5 bits) & shamt (5 bits) & funct (6 bits) & address/const\\
	\hline
	0x23 & 1 & 2 & n/a & n/a & n/a & 0x4\\
	\hline
	35 & \$at & \$v0 & n/a & n/a & n/a & 4\\
	\hline
	100011 & 00001 & 00010 & n/a & n/a & n/a & 0000000000000100\\
	\hline
\end{tabular}
\newline
\vspace{4mm}
\newline
\begin{tabular}{|c|c|c|c|}
	\hline
	Type & Instruction & Binary & Hex\\
	\hline
	I-type & lw \$v0, 4(\$at) & 1000 1100 0010 0010 0000 0000 0000 0100 & 0x8C220004\\
	\hline
\end{tabular}

\section*{2.19.2}
\$t0 = 1010 1010 1010 1010 1010 1010 1010 1010
\newline
\newline
sll \$t2, \$t0, 4 $\rightarrow$ \$t2 = \$t0 $<<$ by 4 bits\\
\$t2 = 1010 1010 1010 1010 1010 1010 1010 0000 = 0xAAAAAAA0\\
\newline
andi \$t2, \$t2, -1 $\rightarrow$ \$t2 = \$t2 \& -1
\newline
\newline
\begin{tabular}{ccccccccc}
	 & 1010 & 1010 & 1010 & 1010 & 1010 & 1010 & 1010 & 0000\\
	\& & 1111 & 1111 & 1111 & 1111 & 1111 & 1111 & 1111 & 1111 \\
	\hline
	& 1010 & 1010 & 1010 & 1010 & 1010 & 1010 & 1010 & 0000\\
\end{tabular}
\newline
\newline
\$t2 = 0xAAAAAAA0

\section*{2.23}
 slt \$t2, \$0, \$t0 \hspace{22mm} \$0 $<$ \$t0 ? \textbf{\$t2 = 1} : \$t2 = 0\\
 bne \$t2, \$0, ELSE \hspace{15mm} go to ELSE if \$t2 $\neq$ 0\\
 j DONE \hspace{35mm}go to DONE\\
 ELSE: addi \$t2, \$t2, 2 \hspace{8mm} \$t2 = \$t2 + 2\\
 DONE:\\
 \newline
 \textbf{\$t2 = 3}
 
 \section*{2.24}
 No, you cannot jump the pc from 0x20000000 to 0x40000000 with a jump instruction because a J-type instruction 
 only allows for a jump address of 26 bits.\\
 \\
 No, you cannot use beq to jump the pc from 0x20000000 to 0x40000000 because it only allows for an address change of 16 bits.\\
 
 \section*{2.26.1}
 \$t1 = 10\\
 \$s2 = 0\\
 LOOP: slt \$t2, \$0, \$t1 \hspace{12mm} \$0 $<$ \$t1 ? \$t2 = 1: \$t2 = 0\\
 \hspace*{15mm}beq \$t2, \$0, DONE  \hspace{4mm}go to DONE if \$t2 == 0\\
 \hspace*{15mm}subi \$t1, \$t1, 1 \hspace{10mm} \$t1 = \$t1 - 1\\
 \hspace*{15mm}addi \$s2, \$s2, 2 \hspace{10mm} \$s2 = \$s2 + 2\\
 j LOOP\\
 DONE:\\
 \newline
 \begin{tabular}{|c|c|c|c|c|c|c|c|c|c|c|}
	 \hline
	 \$t1 & 9 & 8 & 7 & 6 & 5 & 4 & 3 & 2 & 1 & 0\\
	 \hline
	 \$s2 & 2 & 4 & 6 & 8 & 10 & 12 & 14 & 16 & 18 & 20\\
	 \hline
\end{tabular}
\newline
\newline
\textbf{Final Value for \$s2 = 20}

\section*{2.38}
\$t1 = 0x1000 0000\\
\$t2 = 0x1000 0010\\
Data at 0x1000 000 = 0x11223344\\
\newline
lbu \$t0, 0(\$t1)\hspace{10mm}\$t0 = 0x11\\
sw \$t0, 0(\$t2)\hspace{10mm} Address of \$t2 has the data in \$t0 stored there.
\newline
\textbf{\$t2 has the value 0x00000011\\}

\section*{2.40}
Address = 0010 0000 0000 0001 0100 1001 0010 0100 = 0x20014924\\
No, if the PC starts at 0x00000000 it does not have enough bits to change to reach that address.\\
 
\section*{3.1}
5ED4 - 07A4 = 5ED4 + (-07A4)\\
\newline
5ED4 = 0101 1110 1101 0100\\
07A4 = 0000 0111 1010 0100\\
inverted 07A4 = 1111 1000 0011 1011\\
negated 07A4 = 1111 1000 0101 1100\\
\newline
\begin{tabular}{ccccc}
	& 0101 & 1110 & 1101 & 0100\\
	+& 1111 & 1000 & 0101 & 1100\\
	\hline
	& 0101 & 0111 & 0011 & 0000 
\end{tabular}
\newline
\newline
\textbf{5ED4 - 07A4 = 5730}\\

\section*{3.4}
4365 - 3412 = 4365 + (-3412)\\
4362 = 100 011 110 101\\
3412 = 011 100 001 010\\
inverted 3412 = 100 011 110 101\\
negated 3412 = 100 011 110 110\\
\newline
\begin{tabular}{ccccc}
	& 100 & 011 & 110 & 101\\
	+& 100 & 011 & 110 & 110\\
	\hline
	& 000 & 111 & 101 & 011 
\end{tabular}
\newline
\newline
\textbf{4365 - 3412 = 0753}\\

\section*{3.6}
185 - 122 = 185 + (-122)\\
185 = 10111001\\
122 = 01111010\\
inverted 122 = 10000101\\
negated 122 = 10000110\\
\begin{tabular}{cc}
	&10111001\\
	+& 10000110\\
	\hline
	& 00111111 
\end{tabular}
\newline
\newline
\textbf{185 - 122 = 63}
\newline
Neither overflow or underflow occured.

\section*{3.12}
62 * 12\\
62 = 110 010\\
12 = 001 010\\
\begin{tabular}{|c|c|c|c|c|}
	\hline
	Iteration & Step & Mulitplier & Multiplicand & Product \\
	\hline
	0 & Initial Values & 001 010 & 000 110 010 & 000 000 000\\
	\hline
	1 & 1a: 0 $\Rightarrow$ No operation & 001 010 & 000 110 010 & 000 000 000\\
	  & 2: Shift left Multiplicand & 001 010 & 001 100 100 & 000 000 000\\
	  & 3: Shift right Multiplier & 000 101 & 001 100 100 & 000 000 000\\
	\hline
	2 & 1a:  1 $\Rightarrow$ Prod = Prod + Mcand & 000 101 & 001 100 100 & 001 100 100\\
	  & 2: Shift left Multiplicand & 000 101 & 011 001 000 & 001 100 100 \\
	  & 3: Shift right Multiplier & 000 010 & 011 001 000 & 001 100 100\\
	\hline
	3 & 1a: 1 $\Rightarrow$ No operation & 000 010 & 011 001 000 & 001 100 100\\
	  & 2: Shift left Multiplicand & 000 010 & 110 010 000 & 001 100 100\\
	  & 3: Shift right Multiplier & 000 001 & 110 010 000 & 001 100 100\\
	\hline
	4 & 1a: 1 $\Rightarrow$ Prod = Prod + Mcand & 000 001 & 110 010 000 & 111 110 100 \\
	  & 2: Shift left Multiplicand & 000 001 & 001 100 100 000 & 111 110 100  \\
	  & 3: Shift right Multiplier & 000 000 & 001 100 100 000 & 111 110 100 \\
	\hline
\end{tabular}
\newline
\newline
\textbf{62 * 12 = 111 110 100 = 764}
	 
\end{document}