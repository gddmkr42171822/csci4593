\documentclass[12pt]{article}
\linespread{1.3} % 1 and half line spacing
\usepackage[top=1in, bottom=1in, left=.5in, right=.5in]{geometry} % set the margins
\usepackage{helvet} % font type
\usepackage{amsmath} % math equations
\usepackage{graphicx} % resize a table

\begin{document}
	
\title{CSCI 4593: Homework 4}
\author{}
\date{}

\maketitle

\section*{4.4.2}
Add = 70ps\\
Sign extend = 15ps\\
Shift left 2 = 10ps\\
I-Mem = 200ps\\
Mux = 20ps\\
For an unconditional branch we need to first fetch the instruction from instruction memory, sign-extend the lower 16 bits of the instruction,
shift those left by 2, add the result to create the branch target, and finally use a mux to replace the PC with the branch target.\\
\textbf{Cycle time = 200ps + 15ps + 10ps + 70ps + 20ps = 315ps}\\
\subsubsection*{Using the information from Problem 4.4.2, what is the clock cycles time if the only types of instructions we need to support are ALU instructions (ADD, AND, etc.)?}
I-Mem = 200ps\\
Regs = 90ps\\
Mux = 20ps\\
ALU = 90ps\\
In order to support ALU (R-type) instructions, we need instruction memory, registers, 2 muxes, and an ALU.\\
\textbf{Cycle time = 200ps + 90ps + 2}$\mathbf{\times}$\textbf{20ps + 90ps = 400ps}\\


\section*{4.7.4}
Instruction: 1010 1100 0110 0010 0000 0000 0001 0100\\
\\
\scalebox{0.7}{
\begin{tabular}{|c|c|c|c|}
	\hline
	op (Inst[31:26]) & rs (Inst[25:21]) & rt (Inst[20:16]) & address (Inst[15:0])\\
	\hline
	101011 & 00011 & 00010 & 0000000000010100\\
	\hline
	sw & 3 & 2 & 20\\
	\hline
\end{tabular}
}
\newline
\vspace{4mm}
\newline
\scalebox{0.7}{
\begin{tabular}{|c|c|c|c|c|}
	\hline
	Write register (RegDst = X) & ALU (ALUSrc = 1) & Write data (MemtoReg = X) & Branch (Branch = 0) & Jump (Jump = 0)\\
	\hline
	value = 0 (Inst[15-11]) or 2 (Inst[20-16]) & value = 20 (Inst[15-0]) & value = ALU result or Data memory read data & value = PC + 4 & value = PC + 4\\
	\hline
\end{tabular}
}
\section*{4.16.1}
always-taken will hit 3/5 or be 60\% accurate\\
always-not-take will hit 2/5 or be 40\% accurate\\
\section*{4.16.2}
\begin{tabular}{|c|c|c|c|}
	\hline
	Step & Predictor state & Branch state & result\\
	\hline
	1 & not-taken & taken & miss\\
	\hline
	2 & not-taken & not-taken & hit\\
	\hline
	3 & not-taken & taken & miss\\
	\hline
	4 & not-taken & taken & miss\\
	\hline
\end{tabular}
\\
\\
The accuracy is 1/4 or 25\%\\


\section*{4.16.3}
\begin{tabular}{|c|c|c|c|}
	\hline
	Step & Predictor state & Branch state & result\\
	\hline
	1 & not-taken (0) & taken & miss\\
	\hline
	2 & not-taken (1) & not-taken & hit\\
	\hline
	3 & not-taken (0) & taken & miss\\
	\hline
	4 & not-taken (1) & taken & miss\\
	\hline
	5 & taken (3) & not-taken & miss\\
	\hline 
	6 & not-taken (2) & taken & miss\\
	\hline
	7 & taken (3) & not-taken & miss\\
	\hline
	8 & not-taken (2) & taken & miss\\
	\hline
	9 & taken (3) & taken & hit\\
	\hline
	10 & taken (4) & not-taken & miss\\
	\hline
	11 & taken (3) & taken & hit\\
	\hline
	12 & taken (4) & not-taken & miss\\
	\hline
	13 & taken (3) & taken & hit\\
	\hline
	14 & taken (4) & taken & hit\\
	\hline
	15 & taken (4) & not-taken & miss\\
	\hline
\end{tabular}\\
\\
\\
It looks like the long term accuracy will be 3/5 or 60\% when the predictor moves into the predict taken states.  It will always hit on the taken branches and always miss on the not-taken branches.  It will remain in the predict taken states move back and forth.

\end{document}