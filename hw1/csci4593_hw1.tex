\documentclass[12pt]{article}
\linespread{1.3} % 1 and half line spacing
\usepackage[top=1in, bottom=1in, left=.5in, right=.5in]{geometry} % set the margins
\usepackage{helvet} % font type
\usepackage{amsmath} % math equations

\begin{document}
	
\title{CSCI 4593: Homework 1}
\author{}
\date{}

\maketitle

\section*{1.5}
Consider three different processor P1, P2, and P3 executing the same
instruction set.  P1 has a 3GHz clock rate and a CPI of 1.5.  P2 has
a 2.5 GHz clock rate and a CPI of 1.0.  P3 has a 4.0 GHz clock and has
a CPI of 2.2.

\subsection*{1.5(a)}
Which processor has the highest performance expressed in instructions per second?
\[
	\mathrm{P1}\rightarrow \text{CPU time}_\mathrm{P1}=\frac{3*10^9\;\text{cycles/sec}}{1.5\;\text{cycles/sec}} = 2\cdot10^9\;\text{Instructions/sec}
\]
\[
	\mathrm{P2}\rightarrow \text{CPU time}_\mathrm{P2}=\frac{2.5*10^9\;\text{cycles/sec}}{1\;\text{cycles/sec}} = 2.5\cdot10^9\;\text{Instructions/sec}
\]
\[
	\mathrm{P3}\rightarrow \text{CPU time}_\mathrm{P3}=\frac{4*10^9\;\text{cycles/sec}}{2.2\;\text{cycles/sec}} = 1.81\cdot10^9\;\text{Instructions/sec}
\]

\subsection*{1.5(b)}
If the processors each execute a program in 10 seconds, find the number of cycles and the number of instructions.
\begin{gather*}
	\text{IC} = \frac{\text{CPU}\cdot{\text{Clock rate}}}{\text{CPI}}\\
	\text{CPU Clock cycles} = \text{CPI}\cdot{\text{IC}}
\end{gather*}

\[ \text{IC}_{\text{p1}} = \frac{(10\;\text{seconds}*(3*10^9\;\text{cycle/second}))}{1.5\;\text{cycles/instruction}} = 2\cdot{10^{10}}\;\text{instructions}
\]
\[
\text{CPU clock cycle}_{p1} = 1.5\;\text{cycles/instruction}\;*(2*10^{10}\;\text{instructions}) = 3\cdot{10^{10}}\;\text{cycles}
\]
\[ \text{IC}_{\text{p2}} = \frac{(10\;\text{seconds}*(2.5*10^9\;\text{cycle/second}))}{1\;\text{cycles/instruction}} = 2.5\cdot{10^{10}}\;\text{instructions}
\]
\[
\text{CPU clock cycle}_{p2} = 1\;\text{cycles/instruction}\;*(2.5*10^{10}\;\text{instructions}) = 2.5\cdot{10^{10}}\;\text{cycles}
\]
\[ \text{IC}_{\text{p3}} = \frac{(10\;\text{seconds}*(4*10^9\;\text{cycle/second}))}{2.2\;\text{cycles/instruction}} = 1.81\cdot{10^{10}}\;\text{instructions}
\]
\[
\text{CPU clock cycle}_{p3} = 2.2\;\text{cycles/instruction}\;*(1.81*10^{10}\;\text{instructions}) = 4\cdot{10^{10}}\;\text{cycles}
\]

\subsection*{1.5(c)}
We are trying to reduce the execution time by 30\% but this leads to an increase
of 20\% in the CPI.  What clock rate should be have to get this time reduction.
\begin{equation*}
	\text{New clock rate} = \frac{(\text{IC}*(\text{CPI}+\text{CPI}*.2))}{\text{Cpu time}-(\text{Cpu time}*.3)}
\end{equation*}
\begin{align*}
	\text{Clock rate}_{p1} = \frac{(2*10^{10}\;\text{instructions})*(1.8\;\text{cycles/instruction})}{7\;\text{seconds}} = 5.14\cdot10^{9}\;\text{cycles/instruction} \approx 5\;\text{GHz}\\
	\text{Clock rate}_{p2} = \frac{(2.5*10^{10}\;\text{instructions})*(1.2\;\text{cycles/instruction})}{7\;\text{seconds}} = 4.29\cdot10^{9}\;\text{cycles/instruction} \approx 4\;\text{GHz}\\
	\text{Clock rate}_{p3} = \frac{(1.81*10^{10}\;\text{instructions})*(2.64\;\text{cycles/instruction})}{7\;\text{seconds}} = 6.86\cdot10^{9}\;\text{cycles/instruction} \approx 7\;\text{GHz}\\
\end{align*}

\section*{1.6}
Consider two differnt implementations of the same instruction
set architecture.  The instructions can be divided into four classes according to their CPI(class A, B, C, and D).  P1 with a clock rate of 2.5 GHz and CPIs of 1, 2, 3, and 3, and P2 with a clock rate of 3 GHz and CPIs of 2, 2, 2, and 2.\\
\\
Given a program with a dynamic instruction count of 1.0E6 instructions divided
into classes as follows: 10\% class A, 20\% class B, 50\% class C, and 20\% class D,
which implementation is faster?
\begin{gather*}
	\text{CPI} = \frac{\text{Cpu time}*\text{Clock rate}}{\text{Instruction count}}\\
	\text{Cpu time} = \frac{\text{CPI}*{\text{Instruction count}}}{\text{Clock rate}}
\end{gather*}
\begin{center}
\begin{tabular}{|c|c|c|c|}
	\hline
	\multicolumn{4}{|c|}{IC for each instruction class}\\ 
	\hline
	A & B & C & D \\ \hline
	$10^5$ instructions & $2\cdot10^5$ instructions & $5\cdot10^5$ instructions & $2\cdot10^5$ instructions\\ 
	\hline
\end{tabular}
\end{center}

\begin{align*}
	\text{Cpu time}_{p1} = \frac{10^5+2*2*10^5+3*5*10^5+3*2*10^5\;\text{cycles}}{2.5*10^9\;\text{cycles/second}} = 10.4\cdot10^{-4}\;\text{seconds}\\
	\text{Cpu time}_{p2} = \frac{2*10^5+2*2*10^5+2*5*10^5+2*2*10^5\;\text{cycles}}{3*10^9\;\text{cycles/second}} = 6.7\cdot10^{-4}\;\text{seconds}
\end{align*}

\subsection*{1.6(a)}
What is the global CPI for each implementation?
\begin{itemize}
	\item[P1]$\rightarrow\text{CPI}_{p1} = \frac{10.4\cdot{10^{-4}\;\text{seconds}}*2.5\cdot10^9\;\text{cycles/second}}{10^6\;\text{instructions}} = 2.6\;\text{cycles/instruction}$
	\item[P2]$\rightarrow\text{CPI}_{p2} = \frac{6.7\cdot{10^{-4}\;\text{seconds}}*3\cdot10^9\;\text{cycles/second}}{10^6\;\text{instructions}} = 2.0\;\text{cycles/instruction}$
\end{itemize}

\section*{1.8.1}
For each processor find the average capacitive loads.
\begin{equation*}
	\text{Capacitive load} = \frac{\text{Dynamic power}}{\text{Voltage}^2*\text{Clock rate}}
\end{equation*}
Pentium 4 Prescott processor:
\[
	\text{C} = \frac{90\;\text{W}}{(1.25\;\text{V})^2*2.6\;\text{GHz}} = 1.6\cdot{10^{-8}}\;\text{F}
\]
Core I5 Ivy Bridge:
\[
	\text{C} = \frac{40\;\text{W}}{(.9\;\text{V})^2*3.4\;\text{GHz}} = 1.45\cdot{10^{-8}}\;\text{F}
\]

\section*{1.9.1}
Find the total exection time for this program on 1, 2, 4, and 8 
processors, and show the relative speedup of the 2, 4, and 8 processor result relative
to the single processor result.
\begin{gather*}
	\text{Cpu time} = \frac{\text{Instruction count}*\text{CPI}}{\text{Clock rate}}\\
	\text{Speed-up} = \frac{\text{old time}}{\text{new time}}
\end{gather*}
\begin{center}
	\begin{tabular}{|c|c|c|}
		\hline
		OP & CPI & Instruction count \\
		\hline
		arithmetic & 1 & $2.56\cdot{10^9}$ \\
		\hline
		load/store & 12 & $1.28\cdot{10^9}$ \\
		\hline
		branch & 5 & $.256\cdot{10^9}$ \\
		\hline
	\end{tabular}
\end{center}
1 Processor:
	\begin{align*}
	\text{Cpu time}_{1} &= \frac{\frac{1*2.56*10^9}{.7}+\frac{12*1.28*10^9}{.7}+(5*.256*10^9)}{2*10^9} \\
	&= 13.4\;\text{seconds}
	\end{align*}
2 Processors:
	\begin{align*}
	\text{Cpu time}_{2} &= \frac{\frac{1*2.56*10^9}{.7*2}+\frac{12*1.28*10^9}{.7*2}+(5*.256*10^9)}{2*10^9} \\
	&= 7.04\;\text{seconds}
	\end{align*}
	\[\text{Relative speed-up} = \frac{13.4\;\text{seconds}}{7.04\;\text{seconds}} = 1.9
	\]
4 Processors:
	\begin{align*}
	\text{Cpu time}_{4} &= \frac{\frac{1*2.56*10^9}{.7*4}+\frac{12*1.28*10^9}{.7*4}+(5*.256*10^9)}{2*10^9} \\
	&= 3.84\;\text{seconds}
	\end{align*}
	\[\text{Relative speed-up} = \frac{13.4\;\text{seconds}}{3.84\;\text{seconds}} = 3.49
	\]
8 Processors:
	\begin{align*}
	\text{Cpu time}_{8} &= \frac{\frac{1*2.56*10^9}{.7*8}+\frac{12*1.28*10^9}{.7*8}+(5*.256*10^9)}{2*10^9} \\
	&= 2.24\;\text{seconds}
	\end{align*}
	\[\text{Relative speed-up} = \frac{13.4\;\text{seconds}}{2.24\;\text{seconds}} = 5.98
	\]

\section*{1.10}
Assume a 15 cm diameter wafer has a cost of 12, contains 84 dies, and has
0.020 defects/$cm^2$.  Assume a 20 cm diameter wafer has a cost of 15, contains 100
dies, and has 0.031 defects/$cm^2$.

\subsection*{1.10.1}
Find the yield for both wafers.
\begin{align*}
	\text{Yield} = \frac{1}{(1+\frac{\text{Defects per area}*\text{Die area}}{2})^2}
\end{align*}
Wafer 1:
\begin{gather*}
	\text{Wafer Area} = \pi*(\frac{15}{2})^2 = 176.7\;\text{cm}^2\\
	\text{Die area} = \frac{176.7\;\text{cm}^2}{84} = 2.1\;\text{cm}^2\\
	\text{Yield} = \frac{1}{(1+.02\;\text{defects/}\text{cm}^2*\frac{2.1}{2})^2} = .9593
\end{gather*}
Wafer 2:
\begin{gather*}
	\text{Wafer Area} = \pi*10^2 = 314.2\;\text{cm}^2\\
	\text{Die area} = \frac{314.2\;\text{cm}^2}{100} = 3.14\;\text{cm}^2\\
	\text{Yield} = \frac{1}{(1+.031\;\text{defects/}\text{cm}^2*\frac{3.14}{2})^2} = .9082
\end{gather*}

\subsection*{1.10.3}
If the number of dies per wafer is increased by 10\% and the
defects per area unit increases 15\%, find the die area and yield.\\
Wafer 1:
\begin{gather*}
	\text{Die area} = \frac{176.7\;\text{cm}^2}{84+84*.1} = 1.92\;\text{cm}^2\\
	\text{Defects per area} = .02+.02*.15 = .023\\
	\text{Yield} = \frac{1}{(1+.023\;\text{defects/}\text{cm}^2*\frac{1.92}{2})^2} = .9575
\end{gather*}
Wafer 2:
\begin{gather*}
	\text{Die area} = \frac{314.2\;\text{cm}^2}{100+100*.1} = 2.86\;\text{cm}^2\\
	\text{Defects per area} = .031+.031*.15 = .036\\
	\text{Yield} = \frac{1}{(1+.036\;\text{defects/}\text{cm}^2*\frac{2.86}{2})^2} = .9082
\end{gather*}

\section*{1.11}
The results of the SPEC CPU2006 bzip2 benchmark running on a AMD Barcelona has an 
instruction count of 2.389E12, an execution time of 750 s, and a reference time
of 9650 s.

\subsection*{1.11.1}
Find the CPI if the clock cycle time is 0.333 ns.
\[
\text{CPI} = \frac{750}{2.389*10^{12}*.333*10^{-9}} = .94
\]

\subsection*{1.11.2}
Find the SPECratio.
\begin{gather*}
	\text{SPECratio} = \frac{\text{reference time}}{\text{execution time}}\\
	\text{SPECratio} = \frac{9650\;\text{seconds}}{750\;\text{seconds}} = 12.9
\end{gather*}

\subsection*{1.11.4}
Find the increase in CPU time if the number of instructions of the benchmark is increased by 10\% without affecting the CPI.
\begin{gather*}
	\text{Cpu time} = (2.398*10^{12}*1.1)*(.94*1.05)*(.333*10^{-9}) = 866.97\\
	\text{Cpu time increase} = \frac{(866.97-750)}{750} = 15.5\%
\end{gather*}

\subsection*{1.11.6}
Suppose that we are developing a new version of the AMD
Barcelona processor with a 4 GHz clock rate.  We have added some additional
instructions to the instruction set in such a way that the number of instructions
has been reduced by 15\%.  The execution time is reduced to 700 s and the new
SPECration is 13.7.  Find the new CPI.
\[
\text{CPI} = \frac{700*4*10^9}{.85*2.389*10^{12}} = 1.38
\]

\section*{1.12}
Section 1.10 cites as a pitfall the utilization of a subset of the performance
equation as a performance metric.  To illustrate this, consider the following two processors.  P1 has a clock rate of 4 GHz, average CPI of 0.9, and requires the 
execution fo 5.0E9 instructions.  P2 has a clock rate of 3 GHz, an average CPI of
0.75, and requires the execution of 1.0E9 instructions.

\subsection*{1.12.1}
One usual fallacy is to consider the computer with the largest clock rate as having the largest performance.  Check if this is true for P1 and P2.
\begin{gather*}
\text{CPU time (P1)} = \frac{.9*5*10^9}{410^9} = 1.1\;\text{seconds}\\
\text{CPU time (P2)} = \frac{.75*10^9}{3*10^9} = .25\;\text{seconds}
\end{gather*}
P1 has the bigger clock rate but it had worse performance than P2.

\subsection*{1.12.3}
A common fallacy is to use MIPS to compare the performance of two different
processors, and consider that the processor with the largest MIPS has the largest
performance.  Check if this is true for P1 and P2.
\begin{equation*}
\text{MIPS} = \frac{\text{IC}}{\text{CPU time}*10^6}
\end{equation*}
\begin{gather*}
\text{MIPS(P1)} = \frac{5*10^9}{1.1*10^{6}} = 4.5*10^3\\
\text{MIPS(P2)} = \frac{1*10^9}{.25*10^6} = 4*10^3
\end{gather*}
P2 has the better performance but it has the lower MIPS.



\section*{1.13}
Another pitfall cited in Section 1.10 is expecting to improve the overall
performance of a computer by improving only one aspect of the computer.  Consider
a computer running a program that requires 250 s, with 70 s spent executing FP
instructions, 85 s executing L/S instructions, and 40 s spent executing branch
instructions.
\begin{equation}
	\text{Speed-up} = \frac{\text{old time}}{\text{newTime}}
\end{equation}

\subsection*{1.13(a)}
What would be the overall speed-up if the time for FP operation were reduced by 20\%?
oldTime = 250 seconds\\
new FP Speed = $70\times{.80}$ = 56 seconds\\
newTime = $250 - (70 - 56) = 236$\\
Speed-up = $250/236 = 1.059$
\subsection*{1.13(b)}
Using the original execution times, we want to speed-up INT operations. What would the
speed-up factor for INT operations have to be to get an overall speed-up of 1.25?\\
\\
FP = 70 s\\
L/S = 85 s\\
Branch = 40 s\\
INT = 55 s \\
Speed-up = 1.25\\
\[
1.25 = \frac{250}{\text{New CPU time}} \rightarrow 200 = \text{New CPU Time}
\]
\[
200 = 70 + 85 + 40 + \text{INT}_{\text{time}}
\]
$\text{INT}_{\text{time}}$ = 5\\

\section*{1.14}
Assume a program requires the execution of $50\times10^6$ FP instructions,
$110\times10^6$ INT instructions, $80\times10^6$ L/S instructions, and 
$16\times10^6$ branch instructions.  The CPI for each type of instruction is 1, 1, 4, and 2,
respectively.  Assume that the process has a 2 GHz clock rate.
\begin{equation*}
	\text{Cpu time} = \frac{\text{Instruction count}\times{\text{CPI}}}{\text{Clock rate}}
\end{equation*}

\subsection*{1.14(a)}
What would the CPI of FP instructions have to be to get the program to run twice as fast?
\[
\text{Current CPU time} = \frac{(50*10^6*1)+(110*10^6*1)+(80*10^6*4)+(16*10^6*2)}{2*10^9} = .256\;\text{seconds}\\
\]
\[
\frac{\text{Old CPU time}}{\text{New CPU time}} = 2
\]
\[
\text{Old CPU time} = 2*\text{New CPU time}
\]
\[
(50*10^6*\text{CPI}_{\text{fp}})+(110*10^6*1)+(80*10^6*4)+(16*10^6*2) = \frac{\text{Current CPU time}*\text{Clock rate}}{2}
\]
\[
(\text{CPI}_{\text{fp}}*50*10^6) + (462*10^6) = 256*10^6
\]
\[
\text{CPI}_{\text{fp}}*(50*10^6) = 256*10^6 - 462*10^6
\]
\[
\text{CPI}_{\text{fp}} = \frac{-202}{50} < 0 \Rightarrow \text{CPI can't be less than 0 so it's not possible to speed up the program with FP alone}
\]

\subsection*{1.14(b)}
What would the CPI of L/S instructions have to be to get the program to run twice as fast?
\[
	\frac{(50*10^6*1)+(110*10^6*1)+(\text{CPI}_{\text{L/S}}*80*10^6)+(2*16*10^6)}{2*10^9} = .256/2
\]
\[
	\text{CPI}_{\text{L/S}}*80*10^6 = 256*10^6 - 192*10^6
\]
\[
	\text{CPI}_{\text{L/S}} = (256-192)/80 = .8
\]

\end{document}